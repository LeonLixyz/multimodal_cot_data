\documentclass{article}
\usepackage{amsmath}
\usepackage{amssymb}
\usepackage{graphicx}
\usepackage{hyperref}
\usepackage[version=4]{mhchem}

\title{Example 10}
\date{}

\begin{document}
\maketitle

In triangle \(A B C, \angle A=2 \angle B, A B=4\), and \(B C=2 \sqrt{3}\). Find the value of \(A C\).

Solution: 2.
Method 1:\\
Draw the angle bisector \(A D\).\\
\(\triangle A B C \sim \triangle D A C(\angle C A D=\angle A B D\) and \(\angle C=\angle C)\).

\[
\frac{C D}{x}=\frac{x}{2 \sqrt{3}}
\]

By the angle bisector theorem, we have:

\[
\frac{x}{C D}=\frac{4}{2 \sqrt{3}-C D}
\]

Solve for \(x\) using (1) and (2), we get \(x=2\).\\
Method 2:\\
Draw the angle bisector \(A D\).\\
\(\triangle A B C \sim \triangle D A C(\angle C A D=\angle A B D\) and \(\angle C=\angle C)\).

\[
\frac{C D}{x}=\frac{x}{2 \sqrt{3}}=\frac{A D}{4}
\]

We also know that \(\angle D A B=\angle D B A\), and \(A D=D B\).\\
(1) becomes: \(\frac{C D}{x}=\frac{x}{2 \sqrt{3}}=\frac{D B}{4}=\frac{2 \sqrt{3}-C D}{4}=\frac{2 \sqrt{3}}{x+4}\) or \(\frac{x}{2 \sqrt{3}}=\frac{2 \sqrt{3}}{x+4}\)

Solve for \(x\) in (2), we get \(x=2\).\\

\end{document}
